% Options for packages loaded elsewhere
% Options for packages loaded elsewhere
\PassOptionsToPackage{unicode}{hyperref}
\PassOptionsToPackage{hyphens}{url}
\PassOptionsToPackage{dvipsnames,svgnames,x11names}{xcolor}
%
\documentclass[
  letterpaper,
  DIV=11,
  numbers=noendperiod]{scrartcl}
\usepackage{xcolor}
\usepackage{amsmath,amssymb}
\setcounter{secnumdepth}{-\maxdimen} % remove section numbering
\usepackage{iftex}
\ifPDFTeX
  \usepackage[T1]{fontenc}
  \usepackage[utf8]{inputenc}
  \usepackage{textcomp} % provide euro and other symbols
\else % if luatex or xetex
  \usepackage{unicode-math} % this also loads fontspec
  \defaultfontfeatures{Scale=MatchLowercase}
  \defaultfontfeatures[\rmfamily]{Ligatures=TeX,Scale=1}
\fi
\usepackage{lmodern}
\ifPDFTeX\else
  % xetex/luatex font selection
\fi
% Use upquote if available, for straight quotes in verbatim environments
\IfFileExists{upquote.sty}{\usepackage{upquote}}{}
\IfFileExists{microtype.sty}{% use microtype if available
  \usepackage[]{microtype}
  \UseMicrotypeSet[protrusion]{basicmath} % disable protrusion for tt fonts
}{}
\makeatletter
\@ifundefined{KOMAClassName}{% if non-KOMA class
  \IfFileExists{parskip.sty}{%
    \usepackage{parskip}
  }{% else
    \setlength{\parindent}{0pt}
    \setlength{\parskip}{6pt plus 2pt minus 1pt}}
}{% if KOMA class
  \KOMAoptions{parskip=half}}
\makeatother
% Make \paragraph and \subparagraph free-standing
\makeatletter
\ifx\paragraph\undefined\else
  \let\oldparagraph\paragraph
  \renewcommand{\paragraph}{
    \@ifstar
      \xxxParagraphStar
      \xxxParagraphNoStar
  }
  \newcommand{\xxxParagraphStar}[1]{\oldparagraph*{#1}\mbox{}}
  \newcommand{\xxxParagraphNoStar}[1]{\oldparagraph{#1}\mbox{}}
\fi
\ifx\subparagraph\undefined\else
  \let\oldsubparagraph\subparagraph
  \renewcommand{\subparagraph}{
    \@ifstar
      \xxxSubParagraphStar
      \xxxSubParagraphNoStar
  }
  \newcommand{\xxxSubParagraphStar}[1]{\oldsubparagraph*{#1}\mbox{}}
  \newcommand{\xxxSubParagraphNoStar}[1]{\oldsubparagraph{#1}\mbox{}}
\fi
\makeatother


\usepackage{longtable,booktabs,array}
\usepackage{calc} % for calculating minipage widths
% Correct order of tables after \paragraph or \subparagraph
\usepackage{etoolbox}
\makeatletter
\patchcmd\longtable{\par}{\if@noskipsec\mbox{}\fi\par}{}{}
\makeatother
% Allow footnotes in longtable head/foot
\IfFileExists{footnotehyper.sty}{\usepackage{footnotehyper}}{\usepackage{footnote}}
\makesavenoteenv{longtable}
\usepackage{graphicx}
\makeatletter
\newsavebox\pandoc@box
\newcommand*\pandocbounded[1]{% scales image to fit in text height/width
  \sbox\pandoc@box{#1}%
  \Gscale@div\@tempa{\textheight}{\dimexpr\ht\pandoc@box+\dp\pandoc@box\relax}%
  \Gscale@div\@tempb{\linewidth}{\wd\pandoc@box}%
  \ifdim\@tempb\p@<\@tempa\p@\let\@tempa\@tempb\fi% select the smaller of both
  \ifdim\@tempa\p@<\p@\scalebox{\@tempa}{\usebox\pandoc@box}%
  \else\usebox{\pandoc@box}%
  \fi%
}
% Set default figure placement to htbp
\def\fps@figure{htbp}
\makeatother


% definitions for citeproc citations
\NewDocumentCommand\citeproctext{}{}
\NewDocumentCommand\citeproc{mm}{%
  \begingroup\def\citeproctext{#2}\cite{#1}\endgroup}
\makeatletter
 % allow citations to break across lines
 \let\@cite@ofmt\@firstofone
 % avoid brackets around text for \cite:
 \def\@biblabel#1{}
 \def\@cite#1#2{{#1\if@tempswa , #2\fi}}
\makeatother
\newlength{\cslhangindent}
\setlength{\cslhangindent}{1.5em}
\newlength{\csllabelwidth}
\setlength{\csllabelwidth}{3em}
\newenvironment{CSLReferences}[2] % #1 hanging-indent, #2 entry-spacing
 {\begin{list}{}{%
  \setlength{\itemindent}{0pt}
  \setlength{\leftmargin}{0pt}
  \setlength{\parsep}{0pt}
  % turn on hanging indent if param 1 is 1
  \ifodd #1
   \setlength{\leftmargin}{\cslhangindent}
   \setlength{\itemindent}{-1\cslhangindent}
  \fi
  % set entry spacing
  \setlength{\itemsep}{#2\baselineskip}}}
 {\end{list}}
\usepackage{calc}
\newcommand{\CSLBlock}[1]{\hfill\break\parbox[t]{\linewidth}{\strut\ignorespaces#1\strut}}
\newcommand{\CSLLeftMargin}[1]{\parbox[t]{\csllabelwidth}{\strut#1\strut}}
\newcommand{\CSLRightInline}[1]{\parbox[t]{\linewidth - \csllabelwidth}{\strut#1\strut}}
\newcommand{\CSLIndent}[1]{\hspace{\cslhangindent}#1}



\setlength{\emergencystretch}{3em} % prevent overfull lines

\providecommand{\tightlist}{%
  \setlength{\itemsep}{0pt}\setlength{\parskip}{0pt}}



 


\usepackage{booktabs}
\usepackage{longtable}
\usepackage{array}
\usepackage{multirow}
\usepackage{wrapfig}
\usepackage{float}
\usepackage{colortbl}
\usepackage{pdflscape}
\usepackage{tabu}
\usepackage{threeparttable}
\usepackage{threeparttablex}
\usepackage[normalem]{ulem}
\usepackage{makecell}
\usepackage{xcolor}
\KOMAoption{captions}{tableheading}
\usepackage{amsmath}
\usepackage{fontspec}    % Ensure font support
\usepackage{placeins}
\usepackage{float}
\usepackage{setspace}
\usepackage{indentfirst}
\makeatletter
\@ifpackageloaded{caption}{}{\usepackage{caption}}
\AtBeginDocument{%
\ifdefined\contentsname
  \renewcommand*\contentsname{Table of contents}
\else
  \newcommand\contentsname{Table of contents}
\fi
\ifdefined\listfigurename
  \renewcommand*\listfigurename{List of Figures}
\else
  \newcommand\listfigurename{List of Figures}
\fi
\ifdefined\listtablename
  \renewcommand*\listtablename{List of Tables}
\else
  \newcommand\listtablename{List of Tables}
\fi
\ifdefined\figurename
  \renewcommand*\figurename{Figure}
\else
  \newcommand\figurename{Figure}
\fi
\ifdefined\tablename
  \renewcommand*\tablename{Table}
\else
  \newcommand\tablename{Table}
\fi
}
\@ifpackageloaded{float}{}{\usepackage{float}}
\floatstyle{ruled}
\@ifundefined{c@chapter}{\newfloat{codelisting}{h}{lop}}{\newfloat{codelisting}{h}{lop}[chapter]}
\floatname{codelisting}{Listing}
\newcommand*\listoflistings{\listof{codelisting}{List of Listings}}
\makeatother
\makeatletter
\makeatother
\makeatletter
\@ifpackageloaded{caption}{}{\usepackage{caption}}
\@ifpackageloaded{subcaption}{}{\usepackage{subcaption}}
\makeatother
\usepackage{bookmark}
\IfFileExists{xurl.sty}{\usepackage{xurl}}{} % add URL line breaks if available
\urlstyle{same}
\hypersetup{
  colorlinks=true,
  linkcolor={blue},
  filecolor={Maroon},
  citecolor={Blue},
  urlcolor={Blue},
  pdfcreator={LaTeX via pandoc}}


\author{}
\date{}
\begin{document}


\section{The effects of type and token frequency on semantic
extension}\label{the-effects-of-type-and-token-frequency-on-semantic-extension}

\doublespacing

\setlength{\parindent}{4em}

\subsection{Introduction}\label{introduction}

\subsection{Methods}\label{methods}

Following Harmon \& Kapatsinski (2017), two artificial languages were
used: called Dan and Nem (See Figure~\ref{fig-fig1}). In each language,
the same four suffixes were used: -\emph{sil\textsubscript{PL}},
\emph{-dan\textsubscript{PL}}, \emph{-nem\textsubscript{DIM}}, and
-\emph{shoon\textsubscript{DIM}}. Notably, in our language \emph{-dan}
and \emph{-sil} overlap in meaning (they both occur in plural contexts),
and -\emph{nem} and \emph{-shoon} also overlap in meaning (they both
occur in diminutive contexts). Since all four suffixes are possible
candidates for the diminutive plural meaning, we can examine how
properties of the language (mainly type frequency and token frequency)
affect which suffixes are extended to express the diminutive plural
meaning.

\begin{figure}

\centering{

\pandocbounded{\includegraphics[keepaspectratio]{languages.pdf}}

}

\caption{\label{fig-fig1}A description of the suffixes in our artificial
languages. The thicker lines denote the more frequent form in each
language: the plural -\emph{dan\textsubscript{PL}} in the Dan language
and the diminutive -\emph{nem\textsubscript{DIM}} in the Nem language.}

\end{figure}%

During the exposure phase, each suffix was paired with an image. The
suffixes -\emph{sil\textsubscript{PL}} and \emph{-dan\textsubscript{PL}}
were always paired with a picture of multiple large pictures. On the
other hand, the suffixes \emph{-nem\textsubscript{DIM}} and
-\emph{shoon\textsubscript{DIM}} were always paired with a picture of a
single small creature. The design of the stimuli results in participants
being able to learn that -\emph{sil} and \emph{-dan} are either simply
plural or simply non-diminutive. Similarly, \emph{-nem} and
\emph{-shoon} can be learned as either simply singular or simply
diminutive.

Our Experiment comprised of three different conditions (see
Table~\ref{tbl-conditionslist}), one in which the type frequency of the
frequent suffix (i.e., \emph{dan} in Dan and \emph{nem} in Nem) was
higher than the type frequency of the infrequent suffix (by a factor of
4 or 12 vs.~3) while the token frequency of the two suffixes was matched
and set equal to the type frequency of the frequent suffix (both at 12).
We refer to this condition as Type. In the second condition, the token
frequency of the frequent suffix was higher than the token frequency of
the infrequent suffix (by a factor of 4 or 12 vs.~3) while holding the
type frequency of the two suffixes equal to the type frequency of the
infrequent suffix (both at 3). We refer to this condition as Token.
Finally, in the third condition, we manipulated both the type and token
frequency of the frequent suffix such that the token and type frequency
of the frequent suffix were both greater than the token and type
frequency of the infrequent suffix (by a factor of 4 or 12 vs.~3). We
refer to this condition as Type-Token. In this context, a higher type
frequency corresponds to the suffix appearing with a larger number of
distinct stems relative to the competing suffix, while a larger token
frequency corresponds to simply appearing a greater number of times
relative to the competing suffix, regardless of the number of different
stems it occurs with.

\begin{table}

\caption{\label{tbl-conditionslist}Description of each of our
conditions. Note that in Condition 1, there are an equal number of
tokens between the frequent and infrequent items, however there are a
greater number of types in the frequent items. In Condition 2, the
opposite is true: the frequent items occur more, but in the same number
of types as the infrequent items. Finally, in Condition 3, the frequent
items occur both a greater number of times and in a greater number of
different contexts.}

\centering{

\centering\begingroup\fontsize{11}{13}\selectfont

\begin{tabular}{>{\raggedright\arraybackslash}p{12em}rrrr}
\toprule
\multicolumn{1}{c}{ } & \multicolumn{2}{c}{Frequent} & \multicolumn{2}{c}{Infrequent} \\
\cmidrule(l{3pt}r{3pt}){2-3} \cmidrule(l{3pt}r{3pt}){4-5}
\textbf{Condition} & \textbf{Type} & \textbf{Token} & \textbf{Type} & \textbf{Token}\\
\midrule
\textbf{\cellcolor{gray!6}{Type}} & \cellcolor{gray!6}{12} & \cellcolor{gray!6}{12} & \cellcolor{gray!6}{3} & \cellcolor{gray!6}{12}\\
\textbf{Token} & 3 & 12 & 3 & 3\\
\textbf{\cellcolor{gray!6}{Type-Token}} & \cellcolor{gray!6}{12} & \cellcolor{gray!6}{12} & \cellcolor{gray!6}{3} & \cellcolor{gray!6}{3}\\
\bottomrule
\end{tabular}
\endgroup{}

}

\end{table}%

\subsubsection{Procedure}\label{procedure}

Each participant was randomly assigned one of the conditions. In each
condition, participants were first presented with an exposure phase.
After the exposure phase, participants were tested using a production
task, a form choice task, and a comprehension task. We describe each of
these below.\footnote{Additionally, a demo of the experiment can be
  found at the following link:
  \url{https://run.pavlovia.org/znhoughton/generalizability_demo}.}

\paragraph{Exposure Phase}\label{exposure-phase}

Following Harmon \& Kapatsinski (2017), each exposure trial consisted of
the presentation of a picture on the computer screen which was
subsequently followed with a written label for the image as well as an
audio presentation of that label. Specifically, the image first appeared
on the screen and then 1.25 seconds later was followed by both the label
of the creatures on the screen as well as the audio for that creature.
Participants were instructed to type the name of the creature and press
enter. Participants had 4 seconds to respond after the presentation of
the name of the creature and were given feedback as to whether they were
correct or not.

Participants saw each trial within the exposure phase 5 times, each in a
randomized order.

\paragraph{Production Task}\label{production-task}

After the exposure phase, participants were presented with a production
task. In this task, participants were presented with images and told to
produce a label for the image. Specifically, initially the unaffixed
form appeared on the screen along with the corresponding image. 2
seconds later, the four different possible images of that creature
appeared (a singular big creature, multiple big creatures, a single
small creature, and multiple small creatures). Three of these images
disappeared after 1.25 seconds, leaving a single image for the
participant to produce a label for. Participants had 10 seconds to
respond. In the production task, some of the stems were familiar (i.e.,
seen in training) and others were novel (i.e., not seen in training).

Participants saw each trial within the production task 4 times, each in
a randomized order.

\paragraph{Form Choice Task}\label{form-choice-task}

After the production task, participants were presented with a form
choice task. In this task, participants were presented first with the
base, unaffixed form and the corresponding image. 2 seconds later four
images flashed on the screen, remained on the screen for 1.25 seconds,
and disappeared leaving a single image. Along with the single image,
participants were also presented with two possible labels for that
image, one label on the bottom right of the screen and one label on the
bottom left of the screen. Participants were given four seconds to press
either the left arrow or the right arrow to choose the corresponding
label. The labels were counterbalanced with respect to which side of the
screen they appeared on. In the form choice task, some of the stems were
familiar (i.e., seen in training) and others were novel (i.e., not seen
in training). The goal of this task was to assess whether type and/or
token frequency influence the form choice when accessibility differences
between frequent and rare forms have been attenuated.

Participants saw each trial within the form choice task 2 times, each in
a randomized order.

\paragraph{Comprehension Task}\label{comprehension-task}

Finally, participants were presented with a comprehension task. In this
task, participants were first given the label and corresponding audio
for a given creature. After 0.25 seconds, four images appeared on the
screen and participants had 4 seconds to click one of the images on the
screen that corresponded to the label. Similar to the before-mentioned
tasks, in the comprehension task some of the stems were familiar (i.e.,
seen in training) and others were novel (i.e., not seen in training).

Participants saw each trial within the comprehension task 2 times, each
in a randomized order.

\subsection{Analyses and Results}\label{analyses-and-results}

We explore the results for each task in depth in the next
sections.\footnote{All data and code for the analyses can be found here:
  \url{https://github.com/znhoughton/Generalizability-Type-Token}.}

\subsubsection{Production Task}\label{production-task-1}

In order to determine whether the effect of frequency on semantic
extension differed between conditions, we ran a Bayesian linear
mixed-effects model on the production data. The dependent variable was
whether the participant produced the frequent suffix. The frequent
suffix was coded as 1 if participants chose \emph{dan} in the Dan
language or chose \emph{nem} in the Nem lan

We treatment coded condition such that the intercept was the type-token
frequency condition. Thus, a larger intercept indicates that the
frequent suffix was more likely to be produced than the infrequent
suffix when it had a high type and token frequency. A larger coefficient
estimate indicates that the frequent form was more likely to be produced
than the infrequent form in that condition relative to the type-token
frequency condition. We also included meaning and stem novelty as
sum-coded variables. Meaning was a categorical variable with two levels:
original and novel. An original meaning referred to big plural if the
suffix was \emph{dan} or diminutive singular if the suffix was
\emph{nem}. A meaning was novel if it was diminutive plural regardless
of the suffix. stem novelty similarly was a categorical variable with
two levels, familiar or novel. A familiar stem was one that participants
saw in training while a novel stem was one that they did not see in
training. We also included a random intercept for participant and random
slopes for meaning by participant and stem novelty by participant. The
syntax for our model is included below in
Equation~\ref{eq-prodmodelstem}. The results are reported in
Table~\ref{tbl-prodresultsstem} and visualized in
Figure~\ref{fig-prodresultsstem}.

\begin{equation}\phantomsection\label{eq-prodmodelstem}{
\text{frequent\_suffix} \sim \text{condition}*\text{meaning}*\text{stem} + (1 + \text{meaning} * \text{stem} | \text{participant})
}\end{equation}

We find a meaningful effect for the intercept, suggesting that the
frequent suffix is chosen more than the infrequent suffix when there is
a high type and high token frequency. We also find a meaningful effect
for novel stems, suggesting that in general the frequent suffix is
produced more than the infrequent suffix when the stem is novel. Perhaps
most interestingly, we find a meaningful interaction between the type
frequency condition and novel stem, suggesting that in the type
frequency condition, there is a preference for the frequent suffix over
the infrequent suffix when the stem is novel but not when it is
familiar. On the other hand, no such interaction effect is found in the
token frequency condition. Further, we find a three-way interaction
between type frequency condition, novel meaning, and novel stem,
suggesting that learners are more likely to produce a frequent suffix in
the type frequency condition when both the stem and meaning are novel.

Overall, our results suggest that in production, learners generally use
the frequent suffix more regardless of meaning or stem familiarity if
the suffix has both a high type and high token frequency. Additionally,
when the frequent suffix has a higher type frequency than the infrequent
suffix, learners are more likely to use it if the stem is novel, but not
when the stem is familiar. Finally, in general there is no preference to
use the frequent suffix more for novel meanings if the suffix has only a
higher token frequency than the infrequent suffix.

\begin{table}

\caption{\label{tbl-prodresultsstem}Results of the statistical models
for the production task.}

\centering{

\centering\begingroup\fontsize{11}{13}\selectfont

\begin{tabular}{>{\raggedright\arraybackslash}p{20em}rrrrr}
\toprule
\textbf{ } & \textbf{Estimate} & \textbf{Est.Error} & \textbf{Q2.5} & \textbf{Q97.5} & \textbf{\% Samples > 0}\\
\midrule
\textbf{\cellcolor{gray!6}{Intercept (Type-Token Frequency)}} & \cellcolor{gray!6}{1.02} & \cellcolor{gray!6}{0.55} & \cellcolor{gray!6}{-0.05} & \cellcolor{gray!6}{2.14} & \cellcolor{gray!6}{96.91}\\
\textbf{Type Frequency} & -0.74 & 0.71 & -2.20 & 0.58 & 14.57\\
\textbf{\cellcolor{gray!6}{Token Frequency}} & \cellcolor{gray!6}{-0.22} & \cellcolor{gray!6}{0.68} & \cellcolor{gray!6}{-1.59} & \cellcolor{gray!6}{1.12} & \cellcolor{gray!6}{37.24}\\
\textbf{Novel Meaning} & 0.09 & 0.33 & -0.54 & 0.74 & 61.03\\
\textbf{\cellcolor{gray!6}{Novel Stem}} & \cellcolor{gray!6}{0.27} & \cellcolor{gray!6}{0.14} & \cellcolor{gray!6}{-0.01} & \cellcolor{gray!6}{0.56} & \cellcolor{gray!6}{97.06}\\
\addlinespace
\textbf{Type Frequency:Novel Meaning} & 0.28 & 0.42 & -0.55 & 1.12 & 74.46\\
\textbf{\cellcolor{gray!6}{Token Frequency:Novel Meaning}} & \cellcolor{gray!6}{-0.61} & \cellcolor{gray!6}{0.45} & \cellcolor{gray!6}{-1.51} & \cellcolor{gray!6}{0.26} & \cellcolor{gray!6}{8.42}\\
\textbf{Type Frequency:Novel Stem} & 0.36 & 0.17 & 0.02 & 0.70 & 98.11\\
\textbf{\cellcolor{gray!6}{Token Frequency:Novel Stem}} & \cellcolor{gray!6}{-0.29} & \cellcolor{gray!6}{0.21} & \cellcolor{gray!6}{-0.70} & \cellcolor{gray!6}{0.12} & \cellcolor{gray!6}{7.86}\\
\textbf{Novel Meaning:Novel Stem} & -0.01 & 0.13 & -0.26 & 0.25 & 46.68\\
\addlinespace
\textbf{\cellcolor{gray!6}{Type Frequency:Novel Meaning:Novel Stem}} & \cellcolor{gray!6}{0.36} & \cellcolor{gray!6}{0.17} & \cellcolor{gray!6}{0.03} & \cellcolor{gray!6}{0.68} & \cellcolor{gray!6}{98.34}\\
\textbf{Token Frequency:Novel Meaning:Novel Stem} & -0.05 & 0.20 & -0.44 & 0.35 & 40.76\\
\bottomrule
\end{tabular}
\endgroup{}

}

\end{table}%

\begin{figure}

\centering{

\includegraphics[width=1\linewidth,height=\textheight,keepaspectratio]{generalizability-writeup_files/figure-latex/fig-prodresultsstem-1.pdf}

}

\caption{\label{fig-prodresultsstem}Plot of the statistical model
estimates for the production task. The x-axis indicates the condition
(type frequency, token frequency, or type-token frequency). The y-axis
corresponds to the proportion of responses with a frequent suffix. Blue
points indicate that the meaning was novel while purple points
corresponds to the original meaning. The facet indicates whether the
stem was novel or familiar.}

\end{figure}%

\subsubsection{Form Choice Task}\label{form-choice-task-1}

In order to examine the effects of form-choice, we similarly ran a model
analogous to the one we ran for the production task
(Equation~\ref{eq-prodmodelstem}). In the context of the form-choice
task, the dependent variable reflects whether participants chose the
option with the frequent suffix or the one with the infrequent suffix.
Analogous to the first production model, the independent variables were
condition, meaning, and stem novelty. Once again, condition was
treatment coded such that type-token frequency was the reference level
and meaning and stem novelty were sum-coded. The random-effects
structure was identical to that in Equation~\ref{eq-prodmodelstem}. The
model results are presented in Table~\ref{tbl-2afcresultsstem} and
visualized in Figure~\ref{fig-2afcresultsstem}.

\begin{table}

\caption{\label{tbl-2afcresultsstem}Results of the statistical models
for the form-choice task.}

\centering{

\centering\begingroup\fontsize{11}{13}\selectfont

\begin{tabular}{>{\raggedright\arraybackslash}p{20em}rrrrr}
\toprule
\textbf{ } & \textbf{Estimate} & \textbf{Est.Error} & \textbf{Q2.5} & \textbf{Q97.5} & \textbf{\% Samples > 0}\\
\midrule
\textbf{\cellcolor{gray!6}{Intercept (Type-Token Frequency)}} & \cellcolor{gray!6}{-0.01} & \cellcolor{gray!6}{0.07} & \cellcolor{gray!6}{-0.15} & \cellcolor{gray!6}{0.12} & \cellcolor{gray!6}{41.23}\\
\textbf{Type Frequency} & -0.12 & 0.10 & -0.30 & 0.06 & 10.50\\
\textbf{\cellcolor{gray!6}{Token Frequency}} & \cellcolor{gray!6}{0.05} & \cellcolor{gray!6}{0.09} & \cellcolor{gray!6}{-0.13} & \cellcolor{gray!6}{0.22} & \cellcolor{gray!6}{70.71}\\
\textbf{Novel Meaning} & 0.05 & 0.07 & -0.08 & 0.19 & 78.69\\
\textbf{\cellcolor{gray!6}{Novel Stem}} & \cellcolor{gray!6}{-0.01} & \cellcolor{gray!6}{0.07} & \cellcolor{gray!6}{-0.14} & \cellcolor{gray!6}{0.12} & \cellcolor{gray!6}{45.84}\\
\addlinespace
\textbf{Type Frequency:Novel Meaning} & -0.18 & 0.09 & -0.36 & 0.00 & 2.81\\
\textbf{\cellcolor{gray!6}{Token Frequency:Novel Meaning}} & \cellcolor{gray!6}{-0.05} & \cellcolor{gray!6}{0.09} & \cellcolor{gray!6}{-0.22} & \cellcolor{gray!6}{0.13} & \cellcolor{gray!6}{30.07}\\
\textbf{Type Frequency:Novel Stem} & -0.03 & 0.09 & -0.21 & 0.15 & 37.80\\
\textbf{\cellcolor{gray!6}{Token Frequency:Novel Stem}} & \cellcolor{gray!6}{-0.02} & \cellcolor{gray!6}{0.09} & \cellcolor{gray!6}{-0.19} & \cellcolor{gray!6}{0.15} & \cellcolor{gray!6}{41.59}\\
\textbf{Novel Meaning:Novel Stem} & -0.02 & 0.07 & -0.15 & 0.11 & 38.23\\
\addlinespace
\textbf{\cellcolor{gray!6}{Type Frequency:Novel Meaning:Novel Stem}} & \cellcolor{gray!6}{0.12} & \cellcolor{gray!6}{0.09} & \cellcolor{gray!6}{-0.07} & \cellcolor{gray!6}{0.30} & \cellcolor{gray!6}{89.33}\\
\textbf{Token Frequency:Novel Meaning:Novel Stem} & -0.03 & 0.09 & -0.20 & 0.14 & 35.20\\
\bottomrule
\end{tabular}
\endgroup{}

}

\end{table}%

\begin{figure}

\centering{

\includegraphics[width=1\linewidth,height=\textheight,keepaspectratio]{generalizability-writeup_files/figure-latex/fig-2afcresultsstem-1.pdf}

}

\caption{\label{fig-2afcresultsstem}Plot of the statistical model
estimates for the form-choice task. The x-axis indicates the condition
(type frequency, token frequency, or type-token frequency). The y-axis
corresponds to the proportion of responses with a frequent suffix. Blue
points indicate that the meaning was novel while purple points
corresponds to the original meaning. The facet indicates whether the
stem was novel or familiar.}

\end{figure}%

In general we find no meaningful effects except for an interaction
effect between type frequency and novel meaning, suggesting that there
may be a preference for the frequent suffix when it has a higher type
frequency than the infrequent suffix. However, the lack of any other
meaningful effects suggests that when participants are presented with
both possible options, for all conditions (except when there is a novel
meaning in the type frequency condition), participants are equally
likely to choose the frequent suffix as the infrequent suffix (i.e.,
token and type frequency don't seem to effect this).

\subsubsection{Comprehension Task}\label{comprehension-task-1}

In order to examine the results for the comprehension task, similar to
the previous tasks we ran a mixed-effects regression model. The
dependent variable was whether the meaning that participants selected
was novel or original. A positive estimate indicates that participants
chose the novel meaning while a negative estimate indicates that
participants were more likely to choose the original meaning. Unlike the
other tasks, our independent variable was coded quite differently.
Specifically, our conditions varied in what the token-to-type frequency
was. For example, in the Type Frequency condition, while the frequent
suffix occurs 12 times with 12 different types, the infrequent occurs 12
times in only 3 different types. Since this approach maximizes our
power, rather than dividing the data based on whether the suffix was
frequent or infrequent, we instead divided the data based on the
relationship between the type and token frequency of the suffix,
regardless of whether it was the frequent suffix or the infrequent
suffix. More specifically, we divided the data up based on whether the
token-to-type ratio for the form presented on a given trial was 12-12,
12-3, or 3-3. By comparing the 12-3 condition to the 12-12 and 3-3
conditions, we can see how an increase in type frequency and a decrease
in token frequency affect participants' choice of the novel vs.~original
meaning. Thus, hereafter we refer to the 12-3 condition as the Control
condition, the 12-12 condition as Higher Type and the 3-3 condition as
Lower Token.

We treatment coded condition (referred to as condition\_numeric) such
that the intercept was the Control condition. As a result, for each
condition, a positive coefficient indicates that participants are more
likely to choose a novel meaning (relative to the Control
condition/intercept). The model syntax is included below in
Equation~\ref{eq-comprehensionmodel}. Our results for the comprehension
task are included in Table~\ref{tbl-comprehensionresults} and visualized
in Figure~\ref{fig-comprehensionresults}.

\begin{equation}\phantomsection\label{eq-comprehensionmodel}{
\text{meaning} \sim \text{condition\_numeric} + (1 | \text{participant})
}\end{equation}

First, we find a meaningful effect for the Control condition
(Intercept), suggesting that participants are more likely to select the
original meaning when there is a high token-to-type ratio. Additionally,
we find positive coefficient estimates for both Higher Type and Lower
Token conditions suggesting that when an increase in type frequency or
decrease in token frequency is present, participants are more likely to
select the novel meaning than when there is a higher token-to-type
ratio. These results suggest that it may not be type or token frequency
independently that drives entrenchment, but rather the ratio between
type and token frequency. That is, regardless of whether there is an
increase in type frequency, or decrease in token frequency, if the
token-to-type ratio is lower, participants are more likely to select the
novel meaning.

\begin{table}

\caption{\label{tbl-comprehensionresults}Results of the statistical
model for the comprehension task.}

\centering{

\centering\begingroup\fontsize{11}{13}\selectfont

\begin{tabular}{>{\raggedright\arraybackslash}p{12em}rrrrr}
\toprule
\textbf{ } & \textbf{Estimate} & \textbf{Est.Error} & \textbf{Q2.5} & \textbf{Q97.5} & \textbf{\% Samples > 0}\\
\midrule
\textbf{\cellcolor{gray!6}{Intercept (Control)}} & \cellcolor{gray!6}{-2.80} & \cellcolor{gray!6}{0.27} & \cellcolor{gray!6}{-3.35} & \cellcolor{gray!6}{-2.28} & \cellcolor{gray!6}{0}\\
\textbf{Higher Type} & 1.46 & 0.19 & 1.09 & 1.84 & 100\\
\textbf{\cellcolor{gray!6}{Lower Token}} & \cellcolor{gray!6}{1.98} & \cellcolor{gray!6}{0.18} & \cellcolor{gray!6}{1.62} & \cellcolor{gray!6}{2.34} & \cellcolor{gray!6}{100}\\
\bottomrule
\end{tabular}
\endgroup{}

}

\end{table}%

\begin{figure}

\centering{

\includegraphics[width=1\linewidth,height=\textheight,keepaspectratio]{generalizability-writeup_files/figure-latex/fig-comprehensionresults-1.pdf}

}

\caption{\label{fig-comprehensionresults}Plot of the model estimates for
the comprehension task.}

\end{figure}%

Similar to the form-choice task, we also ran a model with stem
familiarity as a fixed-effect. The dependent variable remained the same
(whether participants chose the original or novel meaning), however in
addition to condition, we also included stem novelty (familiar or novel)
as a fixed-effect, along with its interaction with condition. A random
intercept for participant was also included.

The results of the model are presented in
Table~\ref{tbl-comprehensionresultsstem}. Our results show no effect of
stem familiarity on comprehension.

\begin{table}

\caption{\label{tbl-comprehensionresultsstem}Results of the statistical
model for the comprehension task with stem as a fixed-effect.}

\centering{

\centering\begingroup\fontsize{11}{13}\selectfont

\begin{tabular}{>{\raggedright\arraybackslash}p{12em}rrrrr}
\toprule
\textbf{ } & \textbf{Estimate} & \textbf{Est.Error} & \textbf{Q2.5} & \textbf{Q97.5} & \textbf{\% Samples > 0}\\
\midrule
\textbf{\cellcolor{gray!6}{Intercept}} & \cellcolor{gray!6}{-2.81} & \cellcolor{gray!6}{0.28} & \cellcolor{gray!6}{-3.38} & \cellcolor{gray!6}{-2.28} & \cellcolor{gray!6}{0.00}\\
\textbf{12-12} & 1.47 & 0.19 & 1.09 & 1.84 & 100.00\\
\textbf{\cellcolor{gray!6}{3-3}} & \cellcolor{gray!6}{1.98} & \cellcolor{gray!6}{0.18} & \cellcolor{gray!6}{1.62} & \cellcolor{gray!6}{2.35} & \cellcolor{gray!6}{100.00}\\
\textbf{Novel Stem} & 0.02 & 0.12 & -0.21 & 0.25 & 57.66\\
\textbf{\cellcolor{gray!6}{12-12:Novel Stem}} & \cellcolor{gray!6}{-0.03} & \cellcolor{gray!6}{0.16} & \cellcolor{gray!6}{-0.34} & \cellcolor{gray!6}{0.28} & \cellcolor{gray!6}{43.70}\\
\addlinespace
\textbf{3-3:Novel Stem} & -0.11 & 0.16 & -0.41 & 0.20 & 25.02\\
\bottomrule
\end{tabular}
\endgroup{}

}

\end{table}%

Overall, the comprehension results demonstrate that participants are
more likely to select the original meaning when there is a high
token-to-type ratio. However, when the token-to-type ratio is equivalent
(either 12-12 or 3-3) then participants are more likely to select the
novel meaning than when there is a high token-to-type ratio. One caveat
is that participants still prefer the original meaning over the novel
meaning in general (this is evident because adding the upper CI estimate
to the intercept for both conditions still results in a negative
estimate). In other words, while decreasing the token-to-type ratio
decreases the preference for the original meaning, it does not eliminate
it.

\subsection{Rescorla-Wagner Model}\label{rescorla-wagner-model}

In this section we examine whether the Rescorla-Wagner model (Rescorla
\& Wagner, 1972), an associative learning model, can accurately predict
our data or not.

The Rescorla-Wagner model is an associative learning model that has
gained a great deal of popularity. Despite its simplicity, it has been
shown to account for many phenomena in language learning and processing
(Baayen, 2012; Ellis, 2006; Olejarczuk et al., 2018; Ramscar et al.,
2013).

Specifically, the Rescorla-Wagner model is a two-layer feedforward
neural network. It takes as its input a set of cues and predicts an
outcome (Kapatsinski, 2023; Rescorla \& Wagner, 1972). The
Rescorla-Wagner model is defined below in Equation~\ref{eq-rwmodel} and
Equation~\ref{eq-rwmodelabsent}, which differ in whether the predicted
outcome is present or not. The model learns the weight of an association
between a present cue and a present outcome. In the equation,
\(\alpha_C\) denotes the salience of the cue, \(\beta^P_O\) denotes the
salience of an outcome when it is present, and \(\beta^A_O\) denotes the
salience of an outcome when it is absent (Kapatsinski, 2023).
\(\lambda_{max}\) is the learning rate.

\begin{equation}\phantomsection\label{eq-rwmodel}{
\Delta V_{C\rightarrow O } = \alpha_C\beta^P_O(\lambda_{max}-\alpha_O)
}\end{equation}

\begin{equation}\phantomsection\label{eq-rwmodelabsent}{
\Delta V_{C\rightarrow O } = \alpha_C\beta^A_O(\lambda_{min}-\alpha_O)
}\end{equation}

While the Rescorla-Wagner model is often used with a linear activation
function, in the present simulations we use a logistic activation
instead (Equation~\ref{eq-rwlogistic}) . The motivation behind this
decision is that the Rescorla-Wagner model with a logistic activation
function is more sensitive to type frequency than the Rescorla-Wagner
model with a linear activation function (Caballero \& Kapatsinski,
2022). Further, as Kapatsinski (2023) demonstrates, using a logistic
activation function mitigates spurious excitement, a side-effect of
using the linear activation function whereby an absent cue and an absent
outcome become associated with eachother.

\begin{equation}\phantomsection\label{eq-rwlogistic}{
\alpha_O = logit^{-1}(\sum_cV_{C\rightarrow O})
}\end{equation}

Given the Rescorla-Wagner model's success on a wide variety of
linguistic phenomena, it's possible that it is also able to predict the
results we see in this paper. Thus in this section we simulate the
Rescorla-Wagner predictions for our production data in each exposure
condition and compare it to the human data.

For the simulations in this section, \(\alpha\) and \(\beta\) were set
to 0.1, \(\lambda_{min}\) was set to 0, and \(\lambda_{max}\) was set to
2. Our simulation process was as follows: First, for each of the
different exposure conditions, we obtained a matrix of predicted
cue-outcome associations using the Rescorla-Wagner model. This was done
by feeding the model each trial of the exposure phase in a random order.
The cues were the stem identity along with the meaning (e.g.,
bal\_dim\_pl) and the outcome was the suffix that the stem occurred with
on that trial (i.e., \emph{dan}, \emph{nem}, \emph{sil}, or
\emph{shoon}). The model then learned associations between each cue and
each outcome.

Next, for each of the items in the production task, we summed the
associations for each cue-outcome present in that trial. For example,
given the item \emph{baldan} and the meaning big plural, we consulted
the matrix of cue-outcome associations for \emph{bal} and \emph{dan},
big and \emph{dan}, and plural and \emph{dan}. We then summed these to
get the predicted associations for \emph{bal} with the big.pl meaning
and \emph{dan}. This resulted in a matrix that included the model's
learned association strength for any given cue with each of the four
suffixes.

Following this, in order to compare the simulations with the human data,
we calculated association strengths depending on whether the meaning was
original or familiar. For the original meanings, we subtracted the
activation strengths between the original meaning and the appropriate
suffix. Specifically, this is the difference in activation strength
between the meaning cue big plural and the outcome \emph{dan}, and the
meaning cue diminutive singular and the outcome \emph{nem}. If the
language was dan (i.e., \emph{dan} was the frequent suffix), then the
activation strength between diminutive singular and \emph{nem} was
subtracted from the activation strength between big plural and
\emph{dan}. If the language was nem (i.e., \emph{nem} was the frequent
suffix), then the subtraction was in the opposite direction. To
calculate the association strengths for novel meanings, if the frequent
suffix was \emph{dan} then the activation strength between diminutive
plural and \emph{nem} was subtracted from the activation strength
between diminutive plural and \emph{dan}. If the frequent suffix was
\emph{nem} then the opposite was done. This results in a single value
for each trial that is the model's predicted activation of the frequent
suffix relative to the infrequent suffix.

In order to test our model, we evaluated whether the model is a good
predictor of the human data. Thus, we ran two Bayesian logistic
regression models. In each model, the dependent variable was whether the
human learner, for a given trial, selected the frequent suffix or the
infrequent suffix. The first model was a simple model that modeled the
human data as a function of the Rescorla-Wagner activation strength
(referred to as \texttt{freq\_minus\_infreq} in the model syntax), with
random intercepts for participant and stem identity
(Equation~\ref{eq-rwmodel1}). In our second model, we calculated
separately the stem activation and the meaning activation (using the RW
model). The motivation behind this is that the Rescorla-Wagner model is
agnostic to whether the cue is a stem or a meaning (because it
originally has no way of knowing whether a given cue is part of the stem
or the meaning), but participants may show varying sensitivity to one
over the other. Modeling it in this manner allows the statistical model
to fit different slopes for the stem activations and the meaning
activations. We also included fixed-effect for condition as well as an
interaction effect between condition and the stem activations, and an
interaction between the meaning activations and condition
(Equation~\ref{eq-rwmodel3}).

\begin{equation}\phantomsection\label{eq-rwmodel1}{
\text{frequency} \sim \text{freq\_minus\_infreq} + (1 | \text{participant}) + (1|\text{resp\_stem})
}\end{equation}

\begin{equation}\phantomsection\label{eq-rwmodel3}{
\begin{aligned}
\text{frequency} & \sim \text{freq\_minus\_infreq\_stem} + \text{freq\_minus\_infreq\_meaning} + \text{condition} \\ & + \text{freq\_minus\_infreq\_stem:condition} + \text{freq\_minus\_infreq\_meaning:condition} \\ & + (1 | \text{participant}) + (1|\text{resp\_stem})
\end{aligned}
}\end{equation}

The results of the models are presented in Table~\ref{tbl-rwsim1} and
Table~\ref{tbl-rwsim3}.

\begin{table}

\caption{\label{tbl-rwsim1}Results of the statistical model with only
the RW predictions (Equation~\ref{eq-rwmodel1}).}

\centering{

\centering\begingroup\fontsize{11}{13}\selectfont

\begin{tabular}{>{\raggedright\arraybackslash}p{12em}rrrr}
\toprule
\textbf{ } & \textbf{Estimate} & \textbf{Est.Error} & \textbf{Q2.5} & \textbf{Q97.5}\\
\midrule
\textbf{\cellcolor{gray!6}{Intercept}} & \cellcolor{gray!6}{0.34} & \cellcolor{gray!6}{0.31} & \cellcolor{gray!6}{-0.28} & \cellcolor{gray!6}{0.96}\\
\textbf{RW Predictions} & 0.38 & 0.09 & 0.20 & 0.56\\
\bottomrule
\end{tabular}
\endgroup{}

}

\end{table}%

\begin{table}

\caption{\label{tbl-rwsim3}Results of the statistical analysis
containing the separated stem and meaning activations and condition as a
fixed-effect (Equation~\ref{eq-rwmodel3}).}

\centering{

\centering\begingroup\fontsize{11}{13}\selectfont

\begin{tabular}{>{\raggedright\arraybackslash}p{18em}rrrrr}
\toprule
\textbf{ } & \textbf{Estimate} & \textbf{Est.Error} & \textbf{Q2.5} & \textbf{Q97.5} & \textbf{\% Samples > 0}\\
\midrule
\textbf{\cellcolor{gray!6}{Intercept}} & \cellcolor{gray!6}{0.38} & \cellcolor{gray!6}{0.62} & \cellcolor{gray!6}{-0.83} & \cellcolor{gray!6}{1.60} & \cellcolor{gray!6}{73.10}\\
\textbf{Stem Activations} & 0.60 & 0.45 & -0.26 & 1.51 & 91.60\\
\textbf{\cellcolor{gray!6}{Meaning Activations}} & \cellcolor{gray!6}{0.33} & \cellcolor{gray!6}{0.48} & \cellcolor{gray!6}{-0.61} & \cellcolor{gray!6}{1.28} & \cellcolor{gray!6}{75.04}\\
\textbf{Type Frequency} & -0.28 & 0.66 & -1.61 & 0.99 & 33.54\\
\textbf{\cellcolor{gray!6}{Token Frequency}} & \cellcolor{gray!6}{-1.80} & \cellcolor{gray!6}{0.92} & \cellcolor{gray!6}{-3.72} & \cellcolor{gray!6}{-0.16} & \cellcolor{gray!6}{1.45}\\
\addlinespace
\textbf{Stem Activations:Type Frequency} & 0.01 & 0.46 & -0.92 & 0.89 & 51.56\\
\textbf{\cellcolor{gray!6}{Stem Activations:Token Frequency}} & \cellcolor{gray!6}{-0.66} & \cellcolor{gray!6}{0.47} & \cellcolor{gray!6}{-1.61} & \cellcolor{gray!6}{0.25} & \cellcolor{gray!6}{7.88}\\
\textbf{Meaning Activations:Type Frequency} & -0.40 & 1.50 & -3.79 & 2.22 & 40.20\\
\textbf{\cellcolor{gray!6}{Meaning Activations:Token Frequency}} & \cellcolor{gray!6}{2.75} & \cellcolor{gray!6}{0.77} & \cellcolor{gray!6}{1.30} & \cellcolor{gray!6}{4.29} & \cellcolor{gray!6}{99.99}\\
\bottomrule
\end{tabular}
\endgroup{}

}

\end{table}%

Overall we find that the general RW activations for both stem and
meaning are able to account for the human data. Further, we find that
when token frequency is high, meaning activations are a better predictor
(evident by the interaction effect between Meaning Activations and Token
Frequency in Table~\ref{tbl-rwsim3}).

In order to compare which model best fits the data, we performed
leave-one-out cross-validation using the R package \texttt{loo} (Vehtari
et al., 2017). Leave-one-out cross-validation refits the model for each
observation in the dataset, leaving out that observation.\footnote{More
  accurately, Bayesian models are computationally expensive to fit many
  times, so this is approximated using Pareto-smoothed importance
  sampling instead.} For each data point, the expected log predictive
density is calculated. Expected log predictive density is the
log-likelihood of the held-out observation given the model's parameters
(trained without the observation). Each model receives a single expected
log predictive density value by summing the log-likelihood of each
observation using leave-one-out cross-validation. A higher value
indicates the model performs better than the other models. Thus, we
compared the expected log predictive density value for each of these
three models, along with the model of the human data that does not
contain the Rescorla-Wagner predictions.

We performed leave-one-out cross validation for each of the two models
with the RW predictors as well as the model with just the human
predictors. The results of our leave-one-out cross-validation are
included in Table~\ref{tbl-loocv}. When using leave-one-out
cross-validation, a model can be considered as outperforming another
model if the difference in elpd is greater than the standard error of
the elpd. In the case of our models, because all of the elpd differences
are less than the standard error of the elpd, it is difficult to draw
conclusions about which model fits the data best.

\begin{table}

\caption{\label{tbl-loocv}Results of our leave-one-out
cross-validation.}

\centering{

\centering\begingroup\fontsize{11}{13}\selectfont

\begin{tabular}{>{\raggedright\arraybackslash}p{22em}rrrr}
\toprule
\textbf{ } & \textbf{elpd\_diff} & \textbf{se\_diff} & \textbf{elpd\_loo} & \textbf{se\_elpd\_loo}\\
\midrule
\textbf{\cellcolor{gray!6}{Original Statistical Predictors}} & \cellcolor{gray!6}{-0.26} & \cellcolor{gray!6}{0.77} & \cellcolor{gray!6}{-1564.93} & \cellcolor{gray!6}{26.05}\\
\textbf{Stem and Meaning Activations by Condition} & -2.80 & 3.96 & -1567.47 & 25.58\\
\textbf{\cellcolor{gray!6}{Only RW Activation}} & \cellcolor{gray!6}{-18.47} & \cellcolor{gray!6}{6.41} & \cellcolor{gray!6}{-1583.14} & \cellcolor{gray!6}{24.93}\\
\bottomrule
\end{tabular}
\endgroup{}

}

\end{table}%

\subsubsection{Additional Simulations}\label{additional-simulations}

There is evidence that humans learn associations for configural cues
that their parts do not have (Kapatsinski, 2009). For example, in
English syllables, rimes are treated as a single constituent
(Figure~\ref{fig-syllstructure}) and Kapatsinski (2009) demonstrated
that native English speakers can learn to associate rimes with an
outcome even in cases when the onset and nucleus are associated with
different outcomes.

\begin{figure}

\centering{

\includegraphics[width=0.5\linewidth,height=\textheight,keepaspectratio]{English syllable structure.pdf}

}

\caption{\label{fig-syllstructure}A visualization of English syllable
structure.}

\end{figure}%

There is also evidence that adult native English speakers show greater
reliance on semantic cues than phonological cues in production
(Culbertson et al., 2018). For example, Culbertson et al. (2018)
demonstrated that in learning an artificial noun class adults rely more
on semantic cues than phonological cues (whereas interestingly children
rely more on phonological cues than semantic cues).

Thus, in order to further examine the differences between the human and
model predictions we made two modifications to the simulations. First,
instead of modeling the suffix meanings (diminutive plural, diminutive
singular, big plural, big singular) as two separate cues, we also
included a configural cue (analogous to an interaction effect in linear
regression) that could be associated with the outcome. This configural
cue was simply a combination of the two meanings. For example, given the
meaning ``big plural'', the cues comprised ``big'', ``plural'', and
``big.plural''. Similar to the previous simulations, the model was
presented with these cues paired with one of the four suffixes. The
model was trained on the same data that the humans were trained on.

Second, in addition to configural cues, we also increased the alpha
value for semantic cues relative to the phonological cues. This results
in semantic cues being more associable with outcomes than phonological
cues. Specifically, alpha was 0.05 for phonological cues and 0.25 for
semantic cues. All other variables remained unchanged.

Analogous to the previous simulations, we then calculated the model's
predicted activation for the frequent suffix. If the meaning was novel,
this was simply the difference between the activation strength for the
frequent suffix and the activation strength for the infrequent suffix.
If the meaning was original, then it was the difference between
activation strength of big plural with \emph{dan} and the activation
strength of diminutive singular and \emph{nem} if \emph{dan} was the
frequent suffix and vice versa if \emph{nem} was the frequent suffix.

In order to evaluate this simulation, we ran a Bayesian logistic
regression model with the human response as the dependent variable (1 if
they chose the frequent suffix and 0 if they chose the infrequent
suffix) and the model's prediction as the independent variable with
random intercepts for participant and stem. We ran four models, one with
configural cues and modified semantic salience, one with configural
cues, increased semantic salience, and the original statistical
predictors (\(condition*meaning*stem\_condition\)), one with only
configural cues (no modified semantic salience) and one with only
modified semantic salience (no configural cues). We then compared these
with the model with only statistical predictors, and the two models from
the previous simulations (Equation~\ref{eq-rwmodel1} and
Equation~\ref{eq-rwmodel3}).

We then compared these models using leave-one-out-cross-validation
(Table~\ref{tbl-loocv_addsims}).

\begin{table}

\caption{\label{tbl-loocv_addsims}Results of the leave-one-out
cross-validation for the additional simulation. `Modified RW'
corresponds to the model with configural cues and increased saliency for
semantic cues. `RW Model' refers to the RW logistic model. `Statistical
Predictors' refers to the original statistical analysis that contains no
RW model predictions.}

\centering{

\centering\begingroup\fontsize{11}{13}\selectfont

\begin{tabular}{>{\raggedright\arraybackslash}p{22em}rrrr}
\toprule
\textbf{ } & \textbf{elpd\_diff} & \textbf{se\_diff} & \textbf{elpd\_loo} & \textbf{se\_elpd\_loo}\\
\midrule
\textbf{\cellcolor{gray!6}{Configural Cues and Increased Saliency of Semantic Cues}} & \cellcolor{gray!6}{0.00} & \cellcolor{gray!6}{0.00} & \cellcolor{gray!6}{-285.61} & \cellcolor{gray!6}{19.26}\\
\textbf{Configural Cues, Increased Saliency of Semantic Cues, and Original Statistical Predictors} & -0.10 & 1.92 & -285.71 & 19.39\\
\textbf{\cellcolor{gray!6}{Only Increased Saliency of Semantic Cues}} & \cellcolor{gray!6}{-7.03} & \cellcolor{gray!6}{1.88} & \cellcolor{gray!6}{-292.64} & \cellcolor{gray!6}{20.00}\\
\textbf{Only Configural Cues} & -66.86 & 9.89 & -352.48 & 23.73\\
\textbf{\cellcolor{gray!6}{Original Statistical Predictors}} & \cellcolor{gray!6}{-1279.44} & \cellcolor{gray!6}{28.05} & \cellcolor{gray!6}{-1565.05} & \cellcolor{gray!6}{26.02}\\
\addlinespace
\textbf{Stem and Meaning Activations by Condition} & -1281.86 & 27.85 & -1567.47 & 25.58\\
\textbf{\cellcolor{gray!6}{Only RW Activation}} & \cellcolor{gray!6}{-1297.53} & \cellcolor{gray!6}{27.22} & \cellcolor{gray!6}{-1583.14} & \cellcolor{gray!6}{24.93}\\
\bottomrule
\end{tabular}
\endgroup{}

}

\end{table}%

The results of leave-one-out cross-validation suggest that the models
that include configural cues or increased sensitivity to semantic cues
outperform all the original models as well as the model with only
statistical predictors. Among those four, any of the models that include
semantic predictors perform better than the model that includes only
configural cues. In order to visualize this, we also included a
side-by-side plot of the human data with the RW activations from the
model that included both modified semantic salience and configural cues
(Figure~\ref{fig-rwvshumanpreds}). For comparison, we also plotted the
activations from the model without modified semantic salience or
configural cues.

\begin{figure}

\centering{

\includegraphics[width=0.9\linewidth,height=\textheight,keepaspectratio]{generalizability-writeup_files/figure-latex/fig-rwvshumanpreds-1.pdf}

}

\caption{\label{fig-rwvshumanpreds}Plot of the original RW suffix
activations and the modified RW activations (configural cues plus
modified salience of semantic cues) versus the human results. The y-axis
is the probability of producing the frequent meaning. The top facet is
original meanings and the bottom axis is novel meanings. Along the
x-axis, `Modified Sim' corresponds to the RW simulations with configural
cues and increased semantic saliency. `RW Logistic' corresponds to our
original suffix activations. `Human' corresponds to the original human
data. The visualization shows that the original RW suffix activations
seem to fall short in predicting the human data for the token frequency
condition.}

\end{figure}%

Overall, the results of our simulations suggest that an error-driven
associative learning model, specifically the Rescorla-Wagner model, fits
the human data well. Further, a model with configural cues and an
increased salience of semantic cues fits the human data better than even
the original statistical predictors. This parallels previous findings in
demonstrating that humans learn associations for configural cues and
that humans are more sensitive to semantic cues than phonological cues.

\subsection{Discussion}\label{discussion}

Our results suggest that in production,

Our results suggest that in production, having a high type and high
token frequency together lead to a preference for the frequent suffix,
and this preference is even stronger if the stem is novel. On the other
hand, when there is a high type frequency but not a high token
frequency, there is only a preference for the frequent form when the
stem and meaning are both novel. Finally, for token frequency there is
only a preference for the frequent form when the meaning is original.
Overall, these results suggest that type frequency and token frequency
both play a role in semantic extension, with a high type frequency
encouraging semantic extension while a high token frequency encourages
using the suffix to communicate the original meaning. Similar to Harmon
\& Kapatsinski (2017), these effects are mitigated when both forms are
brought into memory.

On the other hand, a similar pattern emerges. When there is a high
token-to-type ratio, the original meaning is preferred more. However,
when there are as many types as there are tokens, there is an increased
preference for the novel meaning relative to when there is a high
token-to-type ratio.

Our results also demonstrate that the Rescorla-Wagner model with a
logistic activation function is able to capture the human data. Further,
the model matches the human data best when it includes configural cues
and an increased salience for semantic cues relative to phonological
cues.

\newpage

\phantomsection\label{refs}
\begin{CSLReferences}{1}{0}
\bibitem[\citeproctext]{ref-baayen2012demythologizingwordfrequency}
Baayen, H. (2012). \emph{Demythologizing the word frequency effect: A
discriminative learning perspective} (G. Libben, G. Jarema, \& C.
Westbury, Eds.; Vol. 47, pp. 171--195). John Benjamins Publishing
Company. \url{https://doi.org/10.1075/bct.47.10baa}

\bibitem[\citeproctext]{ref-caballero2022howagglutinativesearching}
Caballero, G., \& Kapatsinski, V. (2022). How agglutinative? Searching
for cues to meaning in choguita rarámuri (tarahumara) using
discriminative learning. \emph{Morphological Diversity and Linguistic
Cognition}, 121160.
\url{https://www.cambridge.org/core/services/aop-cambridge-core/content/view/E6D16CD11DD783B5AA2B0DEE7C0CC285}

\bibitem[\citeproctext]{ref-culbertson2018childrenprivilegephonological}
Culbertson, J., Jarvinen, H., Haggarty, F., \& Smith, K. (2018).
\emph{Do children privilege phonological cues in noun class learning?}
\emph{40}. \url{https://escholarship.org/uc/item/74g7g684}

\bibitem[\citeproctext]{ref-ellisSelectiveAttentionTransfer2006}
Ellis, N. C. (2006). Selective attention and transfer phenomena in L2
acquisition: Contingency, cue competition, salience, interference,
overshadowing, blocking, and perceptual learning. \emph{Applied
Linguistics}, \emph{27}(2), 164--194.
\url{https://doi.org/10.1093/applin/aml015}

\bibitem[\citeproctext]{ref-harmonPuttingOldTools2017}
Harmon, Z., \& Kapatsinski, V. (2017). Putting old tools to novel uses:
The role of form accessibility in semantic extension. \emph{Cognitive
Psychology}, \emph{98}, 22--44.
\url{https://doi.org/10.1016/j.cogpsych.2017.08.002}

\bibitem[\citeproctext]{ref-kapatsinskiTestingTheoriesLinguistic2009}
Kapatsinski, V. (2009). Testing theories of linguistic constituency with
configural learning: The case of the english syllable. \emph{Language},
\emph{85}(2), 248--277. \url{https://doi.org/10.1353/lan.0.0118}

\bibitem[\citeproctext]{ref-kapatsinskiLearningFastAvoiding2021}
Kapatsinski, V. (2023). Learning fast while avoiding spurious excitement
and overcoming cue competition requires setting unachievable goals:
Reasons for using the logistic activation function in learning to
predict categorical outcomes. \emph{Language, Cognition and
Neuroscience}, \emph{0}(0), 1--22.
\url{https://doi.org/10.1080/23273798.2021.1927120}

\bibitem[\citeproctext]{ref-olejarczukDistributionalLearningErrordriven2018}
Olejarczuk, P., Kapatsinski, V., \& Baayen, R. H. (2018). Distributional
learning is error-driven: The role of surprise in the acquisition of
phonetic categories. \emph{Linguistics Vanguard}, \emph{4}(s2), 1--9.
\url{https://doi.org/10.1515/lingvan-2017-0020}

\bibitem[\citeproctext]{ref-ramscarChildrenValueInformativity2013}
Ramscar, M., Dye, M., \& Klein, J. (2013). Children value informativity
over logic in word learning. \emph{Psychological Science}, \emph{24}(6),
1017--1023. \url{https://doi.org/10.1177/0956797612460691}

\bibitem[\citeproctext]{ref-rescorla1972theorypavlovianconditioning}
Rescorla, R. A., \& Wagner, A. (1972). A theory of pavlovian
conditioning: Variations in the effectiveness of reinforcement and
non-reinforcement. \emph{Classical Conditioning II, Current Research and
Theory}, \emph{Vol. 2}.

\bibitem[\citeproctext]{ref-loopackage}
Vehtari, A., Gelman, A., \& Gabry, J. (2017). Practical bayesian model
evaluation using leave-one-out cross-validation and WAIC.
\emph{Statistics and Computing}, \emph{27}, 1413--1432.
\url{https://doi.org/10.1007/s11222-016-9696-4}

\end{CSLReferences}




\end{document}
